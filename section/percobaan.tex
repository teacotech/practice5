\section{Alat dan Bahan}

Alat dan bahan yang digunakan dalam penentuan massa kritis reaktor ini meliputi\cite{ZUBAIR2024110538}:

\begin{itemize}
    \item \textit{Handling Tool}: Digunakan untuk operasi \textit{loading} (pemasukan) dan \textit{unloading} (pengeluaran) elemen bahan bakar ke dan dari teras reaktor. 
    \item Detektor Neutron (\textit{Fission Chamber}, FC): Detektor proporsional yang sensitif terhadap fluks neutron termal, digunakan sebagai pencacah (sensor) populasi neutron di dalam teras.
    \item Sistem Pencacah (\textit{Counter}/Scaler): Perangkat elektronik yang terhubung ke detektor FC untuk menampilkan dan merekam laju cacah neutron.
    \item Elemen Bahan Bakar: Elemen bahan bakar uranium dengan pengayaan rendah yang akan dimasukkan secara bertahap.
    \item Sumber Neutron: Sumber neutron independen (misalnya Amerisium-Berilium / Am-Be) yang dimasukkan ke teras untuk menghasilkan fluks neutron awal (\(S\)) saat reaktor dalam kondisi subkritis.
    \item Perangkat Lunak Analisis: Program lembar kerja (seperti \textit{Microsoft Excel}) untuk memproses data laju cacah (\(C\)) dan menghitung faktor resiprokal \(1/C\) serta menentukan massa kritis.
\end{itemize}
Spesifikasi bahan bakar reaktor Kartini.
\begin{table}[H]
    \centering
    \begin{tabular}{|p{3cm}|p{3cm}|}\hline
        Parameters & Specification \\ \hline
        Length of Fuel Element & 75 cm\\ \hline
        Outer Diameter & 3.81 cm\\ \hline
        Length of Graphite & 9.4 (Bottom); 6.6 (Top) cm\\ \hline
        Length of UZrH & 38.1 cm\\ \hline
        Diameter of UZrH & 3.56 cm\\ \hline
        Mass of UZrH & 2.235 gr\\ \hline
        Density of UZrH & 5.99 gr/Cm3\\ \hline
        Uranium contents & 8.5\% \\ \hline
        U235 Enrichments & 19.75\% \\ \hline
        Mass of U235 & \(\pm\)38 gr \\ \hline
    \end{tabular}
    \caption{Spesifikasi dari bahan bakar reaktor Kartini.\label{table:1}}
\end{table}
\section{Prosedur Percobaan}
Prosedur eksperimen ini bertujuan mengukur multiplikasi subkritis reaktor untuk memperkirakan massa kritis\cite{Glasstone1994}.

\subsection{Persiapan Kondisi Awal (\textit{Initial Unloading})}
Pada kondisi operasi kritis Reaktor Kartini pada daya 100 kW, jumlah elemen bahan bakar adalah 71 kelongsong dengan massa total \(^{235}\text{U}\) sekitar \(\pm 2645\,\text{gram}\). 
Agar reaktor berada dalam kondisi subkritis total (\textit{zero power}) sebelum pengujian dimulai, dilakukan \textit{unloading} (pengeluaran) elemen bahan bakar sebanyak 8 kelongsong. 
Dengan demikian, jumlah total elemen bahan bakar pada kondisi awal praktikum (\textit{initial condition}) adalah 63 kelongsong. 
Kondisi ini memastikan bahwa faktor multiplikasi efektif (\(K_{\text{eff}}\)) jauh di bawah 1, sehingga regenerasi neutron tidak dapat berlangsung tanpa adanya sumber neutron eksternal.

\subsection{Pemasukan (\textit{Loading}) Bahan Bakar ke dalam Teras Reaktor}

Tujuan utama dari prosedur pemasukan (\textit{loading}) bahan bakar ke dalam teras reaktor adalah untuk menentukan jumlah massa minimum \(^{235}\text{U}\) yang dibutuhkan agar reaktor mencapai kondisi kritis (\(K_{\text{eff}} = 1\))\cite{Glasstone1994}.
Pada kondisi awal, reaktor sudah berisi sejumlah bahan bakar sisa (\textit{initial condition} sebanyak 63 kelongsong, sekitar \(2377\,\text{gram}\) \(^{235}\text{U}\)) dan telah dipasang sumber neutron awal (Am-Be). 
Meskipun demikian, kondisi ini masih subkritis (\(K_{\text{eff}} < 1\)), yang berarti populasi neutron yang ada di dalam teras (termasuk cacah latar belakang) bergantung pada sumber eksternal.
Untuk menentukan (memperkirakan) massa kritis, elemen bahan bakar dimasukkan secara bertahap.
\subsubsection{Pelaksanaan \textit{loading} bahan bakar}
Personil minimal terdiri dari 9 orang:
\begin{enumerate}
    \item Supervisor Reaktor: 1 orang
    \item Operator Reaktor: 2 orang
    \item Operator \textit{Handling Tool}: 1 orang
    \item Operator \textit{Unloading}: 3 orang
    \item Petugas Akuntansi Bahan Nuklir: 1 orang
    \item Petugas Proteksi Radiasi: 1 orang
\end{enumerate}
\par Personil melakukan kegiatan \textit{loading} sesuai prosedur berikut:
\begin{enumerate}
    \item Sebelum percobaan dimulai, yakinlah bahwa telah dilakukan \textit{unloading} (8 buah) bahan bakar, dan catat berat massa bahan bakar yang masih ada dalam teras.
    \item Menaikkan semua batang kendali baik pengaman, kompensasi, dan pengatur berada pada posisi \textit{fully-up}, kemudian lakukan pencacahan neutron (\(C_1\)).
    \item Melakukan \textit{loading} satu buah bahan bakar dan catat berat massa \(^{235}\text{U}\) dalam teras reaktor setelah penambahan, kemudian lakukan pencacahan neutron (\(C_2\)).
          Untuk memperhitungkan (prediksi) massa kritis, tentukan titik potong sumbu ordinat dengan menarik garis dari (\(C_1\) sampai dengan \(C_2\)). Dengan menggunakan \textit{Excel}, hitung prediksi massa kritisnya.\label{cara:3}
    \item Dengan cara yang sama dengan langkah (\ref{cara:3}), lakukan \textit{loading} kembali untuk bahan bakar berikutnya dan lakukan pencacahan neutron berikutnya (\(C_3\)). Kemudian lakukan kembali \text{loading} bahan bakar dan pencacahan (ke-n dengan cacah \(C_n\)) hingga mencapai kritis.
    \item Tentukan massa kritis reaktor dengan membuat grafik \(1/C\) versus massa bahan bakar fissil (\(^{235}\text{U}\)) untuk reaktor Kartini. Kemudian tentukan jari-jari kritis reaktor \(R_c\) dari persamaan (\ref{eq:9}). Bentuk grafik yang diperoleh dalam menuju kondisi kritis bisa bervariasi
          (lurus, cembung atau cekung) seperti ditunjukkan pada gambar (\ref{g:1}).
    \item Kondisi kritis ditandai dengan pernunjukan daya reaktor pada kondisi konstan (\textit{zero power}), atau bahkan terjadi kenaikan daya reaktor secara perlahan dan kontinu.
    \item Apabila indikasi kekritisan telah diperoleh, semua batang kendali diturunkan, kemudian dilanjutkan \textit{loading} bahan bakar yang masih tersisa (apabila masih tersedia).
    \item Setiap penambahan bahan bakar, salah satu batang kendali diturunkan sampai \(0\%\) untuk aspek keamanan.
\end{enumerate}
\subsection{Analisis Hasil Eksperimen}
\begin{figure}[H]
    \centering
    \includegraphics[width=.5\textwidth]{image/image1.png}
    \caption{Contoh grafik antara \(\frac{1}{C}\) vs M (massa kritis \(^{235}\text{U}\)).\label{g:1}}
\end{figure}
Data hasil eksperimen diplot dalam grafik dimana nilai \(\frac{1}{C}\) adalah absis dan massa \(^{235}\text{U}\) sebagai ordinatnya sebagaimana ditampilkan pada gambar (\ref{g:1}).
Dilakukan ekstrapolasi dengan menarik garis ke arah sumbu-x antara dua data berurutan, misal data 1 sampai 2, 2 sampai 3, 3 sampai 4, dan seterusnya. Dengan memperbanyak data akan membuat data semakin konvergen.
Dalam hal ini massa kritis reaktor adalah ketika konvergen saat \(y = 0\).
\par Grafik berbentuk linier adalah yang paling ideal karena ekstrapolasi pada penambahan bahan bakar pada tahap 1 telah dapat memberikan estimasi massa kritis reaktor dengan baik.
Estimasi tahap 1 yang diperoleh kurva cekung memberikan jumlah massa krits yang terlalu kecil, sedang kurva cembung memberikan estimasi yang terlalu besar.
Bentuk kurva yang terlalu cekung umumnya diperoleh bila posisi detektor terlalu jauh dengan sumber neutron, sedang kurva yang telalu cembung umumnya diperoleh apabila posisi detektor terlalu dekat.
Dengan memperbanyak jumlah tahap penambahan bahan bakar, estimasi massa kritis yang diperoleh semakin baik.
\par Pendekatan lain adalah dengan menggunakan \textit{Microsoft Excel}. Dengan cara\label{cara:analisis}:
\begin{enumerate}
    \item Data hasil eksperimen diplot dalam grafik dimana nilai \(\frac{1}{C}\) adalah absis dan massa \(^{235}\text{U}\) sebagai ordinat.
    \item Tambahkan persamaan grafik dalam bentuk \(y = mx+c\).
    \item Karena massa kritis adalah saat \(y = 0\), maka persamaan \(y(0) = mx+c\) dapat diselesaikan deengan \(x=\frac{c}{m}\) dimana nilai x adalah massa kritis \(M_c\).
\end{enumerate}
Jari-jari kritis dapat ditentukan dengan menggunakan nilai massa kritis yang telah diperoleh (\(M_c\)), tinggi aktif bahan bakar (\(H_a\)), dan densitas bahan bakar \(^{235}\text{U}\) (\(\rho\)) menggunakan persamaan (\ref{eq:9}) sebelumnya, didapat
persamaan sebagai berikut.
\begin{align}\label{eq:10}
    R_c^2 = \frac{M_c}{\pi H_a \rho}
\end{align}
Penentuan jari-jari kritis reaktor massa kritis harus konsisten dengan densitas yang digunakan.