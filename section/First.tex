\section{Tujuan Percobaan}
\par Pada praktikum kali ini, peserta praktikum dapat menentukan
jumlah massa kritis dari bahan bakar \(^{235}\text{U}\) yang harus diisikan ke dalam
teras reaktor agar dapat melangsungkan reaksi pembelahan berantai (\textit{chain reaction})
dengan selamat.

\section{Dasar Teori}
\subsection{Kekritisan Reaktor}
\par Pada saat reaktor dioperasikan, terjadi proses reaksi inti bahan bakar \(^{235}\text{U}\) dengan neutron termal yang menghasilkan reaksi pembelahan (fisi)
berantai\cite{Lamarsh2002}. Reaksi fisi \(^{235}\text{U}\) melepaskan neutron baru yang berperan dalam melanjutkan proses ini. Apabila populasi
neutron di dalam teras reaktor pada kondisi ajeg (\textit{steady state}) maka disebut reaktor
dalam kondisi kritis. Dalam hal ini dapat dikatakan bahwa pada saat berlangsungnya proses regenerasi,
jumlah neutron sebelum dan sesudah reaksi fisi \(^{235}\text{U}\) adalah tetap\cite{Glasstone1994}.
Kondisi ini dapat tercapai hanya jika jumlah minimum bahan bakar fisil, yang dikenal sebagai massa kritis, terpenuhi, sehingga
memungkinkan regenerasi neutron dapat berlangsung secara kontinu dengan populasi neutron sebelum dan sesudah
regenerasi adalah sama.
Massa kritis adalah jumlah minimum bahan bakar \(^{235}\text{U}\) yang memungkinkan regenerasi neutron dapat berlangsung secara kontinu, 
sehingga populasi neutron sebelum dan sesudah regenerasi adalah sama. 
\par Didefinisikan bahwa \(K_{eff}\) adalah perbandingan jumlah neutron pada suatu generasi terhadap jumlah neutron pada
generasi sebelumnya, dituliskan sebagai:
\begin{align}\label{eq:1}
  K_{eff} = \frac{\Sigma\,\texttt{neutron pada suatu generasi}}{\Sigma\,\texttt{neutron generasi sebelumnya}}
\end{align}
\begin{itemize}
  \item Bila \(K_{eff} > 1\) maka reaktor disebut dalam kondisi superkritis; dimana populasi neutron di dalam
        teras reaktor pada suatu generasi terus meningkat sebagai fungsi waktu.
  \item Bila \(K_{eff} < 1\), disebut reaktor dalam kondisi subkritis; dimana populasi neutron pada suatu regenerasi terus
        berkurang sebagai fungsi waktu.
  \item Bila \(K_{eff} = 1\), disebut reaktor dalam kondisi kritis, dimana populasi neutron pada suatu regenerasi dalam keadaan
        dalam keadaan setimbang (tetap/ajeg). 
\end{itemize}

\subsection{Faktor yang Memengaruhi Kekritisan Reaktor}

Kekritisan reaktor nuklir dipengaruhi oleh beberapa faktor utama yang menentukan keseimbangan antara produksi neutron fisi dan hilangnya neutron (melalui absorpsi dan kebocoran)\cite{Lamarsh2002}.

\subsubsection{Komposisi Material}
Komposisi material teras reaktor sangat menentukan kekritisan. Faktor utamanya adalah kandungan bahan fisil (seperti \(^{235}\text{U}\) atau \(^{239}\text{Pu}\)) yang memicu reaksi fisi, 
dan bahan fertil (seperti \(^{238}\text{U}\) atau \(^{232}\text{Th}\)) yang dapat dikonversi menjadi bahan fisil baru. Selain itu,
keberadaan material penyerap neutron, seperti boron atau kadmium, dapat menghambat reaksi berantai dengan menghilangkan neutron bebas. 
Kontrol terhadap material penyerap ini sangat penting untuk menjaga kekritisan yang diinginkan\cite{Glasstone1994}.

\subsubsection{Massa dan Densitas}
Kekritisan terkait langsung dengan jumlah bahan fisil. 
Massa bahan fisil harus cukup besar untuk mencapai massa kritis; 
jika massa terlalu kecil, laju kebocoran neutron akan melebihi laju produksi neutron, 
menyebabkan sistem menjadi subkritis. Selain itu, densitas bahan fisil memengaruhi jumlah atom fisil per satuan volume. 
Densitas yang lebih tinggi meningkatkan probabilitas interaksi antara neutron dengan inti fisil, sehingga meningkatkan kekritisan.

\subsubsection{Energi Neutron dan Moderasi}
Probabilitas terjadinya fisi dipengaruhi oleh energi neutron. 
Neutron termal (berenergi rendah) memiliki probabilitas yang jauh lebih tinggi untuk memicu fisi pada bahan seperti \(^{235}\text{U}\), 
sedangkan neutron cepat lebih efektif pada beberapa bahan fisil tertentu, seperti \(^{239}\text{Pu}\). 
Oleh karena itu, moderasi adalah proses penting untuk memperlambat neutron cepat menjadi neutron termal menggunakan material moderator (seperti air ringan, air berat, atau grafit). 
Moderator yang efektif meningkatkan peluang neutron ditangkap oleh bahan fisil, secara langsung meningkatkan \(K_{\text{eff}}\)\cite{Lamarsh2002}.

\subsubsection{Geometri dan Reflektor}
Faktor-faktor ini berkaitan dengan manajemen kebocoran neutron. 
Bentuk geometris material fisil memengaruhi rasio luas permukaan terhadap volume, 
yang secara langsung menentukan laju kebocoran neutron; bentuk yang lebih kompak, seperti bola, akan meminimalkan kebocoran. 
Selain itu, penggunaan reflektor (misalnya, berilium atau grafit) di sekeliling inti reaktor berfungsi memantulkan neutron yang bocor kembali ke dalam inti. 
Penggunaan reflektor secara signifikan mengurangi kebocoran neutron, sehingga meningkatkan efisiensi reaksi dan membantu mencapai kondisi kritis\cite{Glasstone1994}.


\subsection{Penentuan Massa Kritis Reaktor}
\subsubsection{Metode Pendekatan Multiplikasi Subkritis}
Penentuan massa kritis reaktor dapat dilakukan melalui metode eksperimental yang mengamati respons populasi neutron 
terhadap penambahan bahan bakar, yang dikenal sebagai metode multiplikasi subkritis (\textit{subcritical multiplication method})\cite{Lamarsh2002}.
Apabila sumber neutron awal dengan laju produksi \(S\) neutron dimasukkan ke teras, maka pada akhir generasi pertama akan dihasilkan \(K_{\text{eff}} \cdot S\) neutron, 
pada akhir generasi kedua sejumlah \(K_{\text{eff}}^2 \cdot S\), dan seterusnya.

Total perlipatan neutron (\textit{multiplication factor}) di dalam teras \((X)\) adalah penjumlahan dari deret geometris yang dihasilkan:
\begin{align}\label{eq:2}
 X &= \frac{S + K_{\text{eff}}S + K_{\text{eff}}^2S + \dots}{S} = 1 + K_{\text{eff}} + K_{\text{eff}}^2 + \dots
\end{align}
Untuk kondisi subkritis, di mana \(K_{\text{eff}} < 1\), jumlah seluruh neutron yang ada di dalam teras (\(N\)) 
adalah hasil perkalian antara sumber neutron dan faktor multiplikasi \(X\):
\begin{align}\label{eq:3}
     N = S \cdot X = \frac{S}{1 - K_{\text{eff}}}
\end{align}
Jika di area medan neutron teras ditempatkan detektor neutron (misalnya, detektor \textit{Fission Chamber}, FC), 
maka laju cacah yang teramati (\(C\)) adalah sebagian fraksi (\(F\)) dari jumlah neutron total (\(N\)) yang ada di dalam teras:
\begin{align}\label{eq:4}
   C = F \cdot N = \frac{F \cdot S}{1 - K_{\text{eff}}}
\end{align}
Dalam eksperimen penentuan massa kritis, lebih disukai untuk mengamati nilai resiprokal laju cacah, \(1/C\), untuk setiap penambahan bahan bakar\cite{Glasstone1994}: 
\begin{align}\label{eq:5}
\frac{1}{C} = \frac{1 - K_{\text{eff}}}{F \cdot S}
\end{align}
Harga \(K_{\text{eff}}\) akan bertambah seiring pertambahan massa bahan bakar \(^{235}\text{U}\) secara bertahap. 
Bila kondisi kritis telah tercapai (\(K_{\text{eff}}=1\)), maka parameter \(1/C\) dari Persamaan (\ref{eq:5}) akan menjadi nol. 
Massa kritis reaktor kemudian dapat ditentukan dengan menentukan titik nol (\(1/C=0\)) pada grafik korelasi antara \(1/C\) versus massa bahan bakar yang telah dimasukkan.

\subsubsection{Pendekatan Teori Difusi}
Penentuan massa kritis juga dapat dilakukan dengan pendekatan teoritis berdasarkan model reaktor. 
Berdasarkan pendekatan teori difusi satu kelompok untuk reaktor telanjang (\textit{bare reactor}), 
faktor multiplikasi efektif dirumuskan sebagai berikut\cite{Duderstadt1976}:
\begin{align}\label{eq:6}
    K_{\text{eff}} = \frac{K_{\infty}}{1 + M^2B^2}
\end{align}
Di mana \(K_{\infty}\) (faktor multiplikasi tak hingga) dan \(M^2\) (luas migrasi) adalah fungsi dari komposisi material dan dapat dianggap konstan. 
Dengan mensubstitusikan Persamaan (\ref{eq:6}) ke dalam Persamaan (\ref{eq:5}) dan menyederhanakannya, didapatkan:
\begin{align}\label{eq:7}
    \frac{1}{C} \propto 1 - K_{\text{eff}} = 1 - \frac{K_{\infty}}{1 + M^2B^2} = \frac{1 - K_{\infty} + M^2B^2}{1 + M^2B^2}
\end{align}
Dalam kondisi yang mendekati kritis, dapat diasumsikan bahwa \(1/C\) kira-kira linier terhadap \(B^2\) (\textit{buckling}). 
Pada kondisi kritis, \(B^2\) sama dengan \textit{buckling} geometris (\(B_g^2\)), yang untuk teras silinder, dirumuskan:
\begin{align}\label{eq:8}
    B_g^2 = \left(\frac{2.405}{R_{\text{e}}}\right)^2+\left(\frac{\pi}{H_{\text{e}}}\right)^2
\end{align}
Di mana \(R_{\text{e}}\) dan \(H_{\text{e}}\) berturut-turut adalah jari-jari dan tinggi teras terekstrapolasi.

Dengan penambahan bahan bakar secara bertahap, jari-jari teras (\(R\)) akan bertambah, sementara tinggi teras (\(H\)) biasanya tetap. 
Dengan demikian, dapat dibuat grafik korelasi antara \(1/C\) versus \(1/R^2\). 
Harga \(1/C = 0\) berhubungan dengan jari-jari kritis \(R_{\text{c}}\). Massa kritis reaktor (\(M_{\text{c}}\)) selanjutnya dapat ditentukan dari dimensi kritis ini:
\begin{align}\label{eq:9}
    M_{\text{c}} = \pi R_{\text{c}}^2 H_{\text{a}} \rho
\end{align} 
Di mana \(\rho\) adalah densitas bahan bakar (\(\text{g}/\text{cm}^3\)) dan \(H_{\text{a}}\) adalah tinggi aktif teras reaktor.
Secara keseluruhan, penentuan massa kritis dilakukan dengan mengamati pertambahan populasi neutron (cacah) terhadap jumlah gram penambahan bahan bakar ke dalam teras hingga regenerasi neutron mencapai seimbang, 
di mana harga \(K_{\text{eff}} = 1\). Untuk tujuan ini, sumber neutron (misalnya Am-Be) dimasukkan ke dalam teras sebagai sumber awal\cite{Lamarsh2002}.