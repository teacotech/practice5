\section{Data Hasil Percobaan}\label{sec:datapercobaan}
Adapun untuk tampilan \textit{benchmarck core} dari TRIGA sebagai berikut.
\begin{figure}[H]
    \centering
    \includegraphics[width=.4\textwidth]{image/image-fuelpin.png}
    \caption{Bahan bakar yang ditambahkan pada praktikum ini berwarna kuning.\label{data:1}}
\end{figure}

\begin{figure}[H]
    \centering
    \includegraphics[width=.5\textwidth]{image/image-exp1.png}
    \caption{Data Praktikum: ID material, massa, total massa, pencacahan, rerata pencacahan per detik, \(\frac{1}{C}\), dan keterangan kekritisan.\label{data:2}}
\end{figure}

\begin{figure}[H]
    \centering
    \includegraphics[width=.5\textwidth]{image/image-exp2-hasil.png}
    \caption{\(\frac{1}{C}\) versus Penambahan massa bahan bakar.\label{data:3}}
\end{figure}

\section{Pembahasan}
Dari data hasil percobaan(\ref{sec:datapercobaan}) di atas, didapatkan persamaan \(y = -0.0012x+3.1263\). Dari persamaan ini, dengan menggunakan cara yang sama pada bagian(\ref{cara:analisis}),
maka akan didadpatkan massa kritis \(M_c\) sebesar \(M_c = \frac{3.1263}{0.0012} = 2605.25\,\text{gram}\).
\par Kemudian untuk mendapatkan jari-jari kritis, digunakan persamaan(\ref{eq:10}), sehingga didapatkan:
\begin{align*}
    R_c = \sqrt{\left(\frac{M_c}{\pi H_a \rho}\right)} = \sqrt{\left(\frac{2605,25}{\pi (38.1)(5.99)}\right)}=1.906\,\text{cm}
\end{align*}
Dari sini, \(R_c\) dan \(M_c\) perhitungan menggunakan persamaan di atas, adalah \textit{Bulk Geometric Property}, dengan catatan nilai dari 
nilai \(R_c\) yang dihitung hitung adalah Radius Kritis teoretis untuk model homogen yang disederhanakan, yang memiliki sifat neutronik (kekritisan) yang sama dengan inti heterogen yang sebenarnya. 
Nilai ini membantu memvalidasi nilai \(M_c\) eksperimental terhadap model fisika reaktor dasar, yang sering kali memperlakukan material inti sebagai campuran homogen tunggal.
Jika penambahan massa total untuk sampai kritis pada tabel(\ref{data:1}), massa total sebesar \(255.4646\,\text{gram}\), dalam hal ini, teras reaktor perlu
diisi bahan bakar sampai mencapai massa kritis, sebesar katakanlah \(\Delta M = 2605.25-255.4646 = 45.7854\,\text{gram}\). 
Dengan melihat tabel spesifikasi(\ref{table:1}), rerata massa dari bahan bakar \(^{235}\text{U}\) sebesar 38 gram, sehingga dibutuhkan paling tidak penambahan dua kali bahan bakar, sehingga 
menjadi 71 buah bahan bakar dari hasil praktikum kali ini. Namun, jika melihat pada tabel, tanpa ekstrapolasi, kita dapat melihat pada(\ref{data:2}) bahwa \(\frac{1}{C} \approx 0\) ketika bahan bakar
total sebanyak 69.
\par Dari data yang disajikan pada \textit{PROSIDING SEMINAR PENELITIAN DAN PENGELOLAAN PERANGKAT NUKLIR--BATAN}, dapat dilihat untuk operasi reaktor daya 100 kW untuk 360 jam/tahun, atau 2 kali lipatnya, keterangan
kritis dari reaktor Kartini tercapai dengan kisaran jumlah bahan bakar ketika menyentuh angka 69, 70, dan 71 buah\cite{Mudjilan2013Analisis}.

\section{Kesimpulan}
Pada praktikum kali ini, telah didapat data untuk massa kritis reaktor kartini, pada \textit{Practice-5 Kekritisan Reaktor} tanggal 28 November 2025, sebesar
\(M_{c-extra}= 2605.25\,\text{gram}\). Namun, dengan melihat tabel(\ref{data:1}), dapat dilihat pada massa total bahan bakar untuk sampai pada keadaan reaktor kritis, sebesar \(M_{c-tabel}=255.4646\,\text{gram}\).
Dapat dihitung error relatif nilai \(M_c\) sebesar \(err_{rel} = \frac{M_{c-extra}-M_{c-tabel}}{M_{c-extra}} = 2\%\), dapat dituliskan
\(M_c = 2605.25\pm 0.02\). Kemudian dengan jari-jari kritis untuk mencapai massa kritis reaktor sebesar \(R_c=1.906\,\text{cm}\), dengan keterangan
bahwa nilai ini merupakan \textit{Bulk Geometry Property}, dan dikarenakan \(M_c\) merupakan sifat neutronik (\textit{neutronic property}), maka perlu diperhitungkan metode lain seperti penggunaan software \textit{MCNP}, dan software neutronik lain
untuk mendapatkan nilai teoritis sehingga pembanding error di atas lebih valid.